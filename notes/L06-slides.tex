\documentclass{beamer}

% TODO: print out https://www.fuzzingbook.org/code/Intro_Testing.py

\usetheme{Boadilla}

%\includeonlyframes{current}

\usepackage{times}
\usefonttheme{structurebold}
\usepackage{listings}
\usepackage[listings]{tcolorbox}

\usepackage{pgf}
\usepackage{tikz}
\usepackage{alltt}
\usepackage[normalem]{ulem}
\usetikzlibrary{arrows}
\usetikzlibrary{automata}
\usetikzlibrary{shapes}
\usetikzlibrary{backgrounds}
\usepackage{amsmath,amssymb}
\usepackage{rotating}
\usepackage{ulem}
\usepackage{pythonhighlight}

\usetikzlibrary{arrows,automata,shapes}
\tikzstyle{block} = [rectangle, draw, fill=blue!20, 
    text width=5em, text centered, rounded corners, minimum height=2em]
\tikzstyle{bt} = [rectangle, draw, fill=blue!20, 
    text width=4em, text centered, rounded corners, minimum height=2em]

\lstdefinelanguage{JavaScript}{
  keywords={typeof, new, true, false, catch, function, return, null, catch, switch, var, if, in, while, 
do, else, case, break},
  keywordstyle=\color{blue}\bfseries,
  ndkeywords={class, export, boolean, throw, implements, import, this},
  ndkeywordstyle=\color{darkgray}\bfseries,
  identifierstyle=\color{black},
  sensitive=false,
  comment=[l]{//},
  morecomment=[s]{/*}{*/},
  commentstyle=\color{purple}\ttfamily,
  stringstyle=\color{red}\ttfamily,
  morestring=[b]',
  morestring=[b]''
}

%\setbeamercovered{dynamic}
\setbeamertemplate{footline}[page number]{}
\setbeamertemplate{navigation symbols}{}
\usefonttheme{structurebold}

\title{Software Testing, Quality Assurance \& Maintenance---Lecture 6}
\author{Patrick Lam\\University of Waterloo}
\date{January 23, 2026}

\colorlet{redshaded}{red!25!bg}
\colorlet{shaded}{black!25!bg}
\colorlet{shadedshaded}{black!10!bg}
\colorlet{blackshaded}{black!40!bg}

\colorlet{darkred}{red!80!black}
\colorlet{darkblue}{blue!80!black}
\colorlet{darkgreen}{green!80!black}

\newcommand{\rot}[1]{\rotatebox{90}{\mbox{#1}}}
\newcommand{\gray}[1]{\mbox{#1}}

\newenvironment{changemargin}[1]{% 
  \begin{list}{}{% 
    \setlength{\topsep}{0pt}% 
    \setlength{\leftmargin}{#1}% 
    \setlength{\rightmargin}{1em}
    \setlength{\listparindent}{\parindent}% 
    \setlength{\itemindent}{\parindent}% 
    \setlength{\parsep}{\parskip}% 
  }% 
  \item[]}{\end{list}}



\begin{document}

\usebackgroundtemplate{\tikz\node[opacity=0.1]{\includegraphics[width=\paperwidth]{L06/06919_chalk_v1.jpg}};}
\begin{frame}
  \titlepage
  {\scriptsize
    Background: schist---metamorphic rock.
  }
\end{frame}

\usebackgroundtemplate{}
\part{Metamorphic Testing}
\begin{frame}
  \partpage
  
\end{frame}

\usebackgroundtemplate{\tikz\node[opacity=0.1]{\includegraphics[width=\paperwidth]{L05/960px-Delphi_Temple_of_Apollo.jpeg}};}
\begin{frame}
  \frametitle{No Oracle? Now what?}
  \Large
  \begin{changemargin}{2em}
    Want to run zillions of tests. \\
    What's the expected output?\\[1em]

    We had implicit oracles, e.g. ``doesn't crash''.\\[1em]

    Can we do better?
  \end{changemargin}
\end{frame}
\usebackgroundtemplate{}

\begin{frame}[fragile]
  \frametitle{Consider min}
  \Large
  \begin{changemargin}{2em}
    Suspend disbelief \& pretend we have no oracle:
\begin{python}
def min(a,b):
    if a < b:
        return a
    else:
        return b
\end{python}
    How can we generate heaps of tests?
  \end{changemargin}
\end{frame}

\begin{frame}[fragile]
  \frametitle{``New tests from old''}
  \begin{center}
    \includegraphics[width=.4\textwidth]{L06/20220802_042715719_lamp.jpg}
  \end{center}
\end{frame}

\begin{frame}
  \frametitle{``New tests from old'' II}

  \Huge
  \begin{changemargin}{2em}
    Say \texttt{min(3, 5) = X}.\\[2em]
    Can we construct another test case with known output?
  \end{changemargin}
\end{frame}

\begin{frame}
  \frametitle{``New tests from old'' III}

\Large
  \begin{changemargin}{2em}
    \texttt{min(5, 3) = X} also.\\[2em]
    A simple example of metamorphic testing:\\
    even if you don't know X, it's the same X:\\[1em]
    \texttt{min(5, 3) = min(3, 5)}.\\[1em]
    Always domain-specific.
  \end{changemargin}
\end{frame}


\begin{frame}
  \frametitle{Observations on metamorphic testing}
  \Large
  \begin{changemargin}{1em}
    You are testing $f$ and you have input $x_0$.
    \begin{enumerate}
    \item given one input $x_0$, generate another input $x_1$;
    \item you know some property of $f(x_1)$ in relation to $f(x_0)$;
    \item you don't necessarily know $f(x_1)$;
      \item in fact, you don't need to know $f(x_0)$ in advance!
    \end{enumerate}
  \end{changemargin}
\end{frame}

\begin{frame}
  \frametitle{min and metamorphic testing: no outputs}
  \Large
  \begin{changemargin}{2em}
    You can randomly generate 1 zillion test inputs for \texttt{min}.\\[1em]
    By our assumption, we don't have the corresponding outputs, \\ just that \texttt{min} shouldn't crash.
  \end{changemargin}
\end{frame}

\begin{frame}
  \frametitle{min and metamorphic testing: what we have}
  \Large
  \begin{changemargin}{2em}
    \begin{itemize}
    \item can generate another 1 zillion test cases, by inverting parameter order.
    \item also don't have outputs for these cases, but
    \item we know that it should be the same as the uninverted input!
    \end{itemize}
    That's something, which is better than nothing.
  \end{changemargin}
\end{frame}

\begin{frame}
  \frametitle{Another example: search}
  \includegraphics[width=\textwidth]{L06/duckduckgo.png}
\end{frame}

\begin{frame}
  \frametitle{Exclusion}
  \Large
  \begin{changemargin}{2em}
    \begin{enumerate}
    \item Search for ``metamorphic''.
    \item Get, say, 3300 results.
    \item Search for ``metamorphic -testing \\ \qquad (i.e. exclude tesing).
    \item Now get 4200 results.
    \item ???
    \end{enumerate}
    ~\\
    How do we use this insight to get new tests from old?
  \end{changemargin}
\end{frame}
  
\begin{frame}
  \frametitle{Another example: text-to-speech}
  \begin{center}
    \includegraphics[width=.6\textwidth]{L06/20260119_022953163_microphone.jpg}
  \end{center}
\end{frame}

\begin{frame}
  \frametitle{Speech-to-text scenario}
  \Large
  \begin{changemargin}{2em}
    You are writing a speech-to-text processor.\\[1em]
  \begin{center}
    \includegraphics[width=.2\textwidth]{L06/20260119_022953163_microphone.jpg}
  \end{center}
    Input: audio file.\\
    Output: text.
  \end{changemargin}
\end{frame}

\begin{frame}
  \frametitle{Testing speech-to-text}
  \Large
  \begin{changemargin}{2em}
    \begin{enumerate}
    \item manual test generation: you record a text and compare to known good output (oracle).
    \item you (or I) get 130 students to do that.
    \item you generate audio using text-to-speech.
    \item you use Mechanical Turk to get audio.
    \end{enumerate}
    ~\\
    Still not even close to covering the space of valid inputs, e.g. accents.
  \end{changemargin}
\end{frame}

\begin{frame}
  \frametitle{Getting lots of inputs\ldots}
  \Large
  \begin{changemargin}{2em}
    Well, there are a lot of audio files on the Internet\ldots \\[2em]
    \ldots but there is no ground truth, and human oracles are expensive.
  \end{changemargin}
\end{frame}

\begin{frame}
  \frametitle{What about property testing?}
  \Large
  \begin{changemargin}{2em}
    ``It doesn't crash on any input.'' OK\ldots\\
    ``It doesn't transcribe acoustic music into words.'' I guess?\\[2em]
    These properties don't really validate the system.
  \end{changemargin}
\end{frame}

\begin{frame}
  \frametitle{Transforming inputs}
  \Large
  \begin{changemargin}{2em}
    Let's say you have one input, which transcribes to \texttt{out}. You can:
    \begin{itemize}
    \item double the volume; or, 
    \item raise the pitch; or,
    \item increase the tempo; or,
    \item add background static; or,
    \item add traffic noises; or,
    \item combine any of these.
    \end{itemize}
  \end{changemargin}
\end{frame}

\begin{frame}
  \frametitle{Implications of transforming inputs}
  \Large
  \begin{changemargin}{2em}
    ``add traffic noises'' has a lot of freedom; \\
    you can add 10 different traffic noises.\\[1em]
    Then, double the volume on each of them;\\
    now have 22 test cases.\\[1em]
    etc.\\[1em]
  \end{changemargin}
\end{frame}

\begin{frame}
  \frametitle{Further implications}
  \Large
  \begin{changemargin}{2em}
    What's more, you know the output for all these cases.\\
    
    Your tests cover much more of the input space.\\
    Still no accents, though.\\[1em]
    What if you didn't have the known output?\\
    Can download any audio from the Internet.\\
    Still have the equality relation with the transformed inputs.
  \end{changemargin}
\end{frame}

\begin{frame}
  \frametitle{Example: YouTube search}
  \includegraphics[width=\textwidth]{L06/metamorphic-youtube.png}
\end{frame}

\begin{frame}
  \frametitle{Example: tagged image search}
  \includegraphics[width=\textwidth]{L06/fauna.png}
\end{frame}

\begin{frame}
  \frametitle{Example: tagged image search}
  \Large
  \begin{changemargin}{2em}
    For our example: tags ``red'' and ``blue''.\\
    An image may be tagged both ``red'' and ``blue''.\\[1em]

    Say there are 4 images. \\
    \{1, 2, 3\} are tagged ``red''.\\
    \{1, 2, 4\} are tagged ``blue''.
  \end{changemargin}
\end{frame}

\begin{frame}
  \frametitle{Some tagged image search queries}
  \Large
  \begin{changemargin}{2em}
    Query: ``has any tag'': \{ 1, 2, 3, 4\} (n=4).\\
    Query: ``red'': \{1, 2, 3\} (n=3).\\
    Query: ``blue'': \{1, 2, 4\} (n=3).\\[1em]
    It must be that
    \[ 3 + 3~ (red + blue) \ge 4 ~(any). \]
  \end{changemargin}
\end{frame}

\begin{frame}
  \frametitle{Metamorphic relation output patterns}
  \Large
  \begin{changemargin}{2em}
    A list of patterns from the paper:
    \begin{itemize}
    \item equivalence: same output items, perhaps in different order;
    \item equality: same output items, same order;
    \item subset: follow-up output has a subset of original output;
    \item disjoint: source and follow-up outputs have no items in common;
    \item complete: union of follow-up outputs completely make up source output;
    \item difference: source and follow-up outputs differ by a specific set $D$.
    \end{itemize}
  \end{changemargin}
\end{frame}

\begin{frame}
  \frametitle{Metamorphic relation output pattern: equivalence}
  \Large
  \begin{changemargin}{2em}
    Source input is a query.\\
    Follow-up input is the same query but with some different ordering requested, like ``sort by date''.\\[1em]
    Expect: same items in outputs, but different order.
  \end{changemargin}
\end{frame}

\begin{frame}
  \frametitle{Metamorphic relation output pattern: equality}
  \Large
  \begin{changemargin}{2em}
    Had this in the \texttt{min} example at start; also speech-to-text.\\[1em]

    Another example: \\
    \begin{changemargin}{1em}
    Source input is a query.\\
    Follow-up input is the same query but explicitly requesting the default ordering (e.g. sort by relevance).\\[1em]
    Expect: same items in outputs, with same order.
    \end{changemargin}
  \end{changemargin}
\end{frame}

\begin{frame}
  \frametitle{Metamorphic relation output pattern: subset}
  \Large
  \begin{changemargin}{2em}
    Search engine example with exclusions was like this.\\[1em]
    Another example: \\
    \begin{changemargin}{1em}
    Source input is a geolocation query with radius of 50km.\\
    Follow-up input is the same query but a smaller radius.\\[1em]
    Expect: follow-up output is subset of source output.
    \end{changemargin}
  \end{changemargin}
\end{frame}

\begin{frame}
  \frametitle{Metamorphic relation output pattern: disjoint}
  \Large
  \begin{changemargin}{2em}
    Example: \\
    \begin{changemargin}{1em}
    source input = Spotify albums of ``michael buble'' from 2012,\\
 follow-up input = Spotify albums of ``michael buble'' from 2014. \\[1em]
Expect: there should be no items in both the source and the follow-up output sets.
    \end{changemargin}
  \end{changemargin}
\end{frame}

\begin{frame}
  \frametitle{Metamorphic relation output pattern: complete}
  \Large
  \begin{changemargin}{2em}
    This is a relation between the source output and the set of follow-up outputs.\\[1em]

    Example context: There are short, medium, and long YouTube
    videos. \\[1em]

    If the source input is for keyword ``testing'', then one can make three follow-up inputs: short ``testing'', medium ``testing'', and long ``testing''.\\[1em]
    
    Expect: bombined, the results for short, medium, and long ``testing'' videos should be the same as the results for just ``testing''.
  \end{changemargin}
\end{frame}

\begin{frame}
  \frametitle{Metamorphic relation output pattern: complete}
  \Large
  \begin{changemargin}{2em}
    Example:
    \begin{changemargin}{1em}
    I upload two videos which I know to be similar except for the length and title. \\
    The create operation returns the video uploaded.\\[1em]
    Expect: the source and follow-up metadata only differ on length and title,\\
    $\qquad$ other properties may be the same.
    \end{changemargin}
  \end{changemargin}
\end{frame}

\part{Code Review}
\begin{frame}
  \partpage
\end{frame}

\end{document}
