\documentclass{beamer}

% TODO: print out https://www.fuzzingbook.org/code/Intro_Testing.py

\usetheme{Boadilla}

%\includeonlyframes{current}

\usepackage{times}
\usefonttheme{structurebold}
\usepackage{listings}

\usepackage{pgf}
\usepackage{tikz}
\usepackage{alltt}
\usepackage[normalem]{ulem}
\usetikzlibrary{arrows}
\usetikzlibrary{automata}
\usetikzlibrary{shapes}
\usetikzlibrary{backgrounds}
\usepackage{amsmath,amssymb}
\usepackage{rotating}
\usepackage{ulem}

\usetikzlibrary{arrows,automata,shapes}
\tikzstyle{block} = [rectangle, draw, fill=blue!20, 
    text width=5em, text centered, rounded corners, minimum height=2em]
\tikzstyle{bt} = [rectangle, draw, fill=blue!20, 
    text width=4em, text centered, rounded corners, minimum height=2em]

\lstdefinelanguage{JavaScript}{
  keywords={typeof, new, true, false, catch, function, return, null, catch, switch, var, if, in, while, 
do, else, case, break},
  keywordstyle=\color{blue}\bfseries,
  ndkeywords={class, export, boolean, throw, implements, import, this},
  ndkeywordstyle=\color{darkgray}\bfseries,
  identifierstyle=\color{black},
  sensitive=false,
  comment=[l]{//},
  morecomment=[s]{/*}{*/},
  commentstyle=\color{purple}\ttfamily,
  stringstyle=\color{red}\ttfamily,
  morestring=[b]',
  morestring=[b]''
}

%\setbeamercovered{dynamic}
\setbeamertemplate{footline}[page number]{}
\setbeamertemplate{navigation symbols}{}
\usefonttheme{structurebold}

\title{Software Testing, Quality Assurance \& Maintenance---Lecture 5}
\author{Patrick Lam\\University of Waterloo}
\date{January 19, 2026}

\colorlet{redshaded}{red!25!bg}
\colorlet{shaded}{black!25!bg}
\colorlet{shadedshaded}{black!10!bg}
\colorlet{blackshaded}{black!40!bg}

\colorlet{darkred}{red!80!black}
\colorlet{darkblue}{blue!80!black}
\colorlet{darkgreen}{green!80!black}

\newcommand{\rot}[1]{\rotatebox{90}{\mbox{#1}}}
\newcommand{\gray}[1]{\mbox{#1}}

\newenvironment{changemargin}[1]{% 
  \begin{list}{}{% 
    \setlength{\topsep}{0pt}% 
    \setlength{\leftmargin}{#1}% 
    \setlength{\rightmargin}{1em}
    \setlength{\listparindent}{\parindent}% 
    \setlength{\itemindent}{\parindent}% 
    \setlength{\parsep}{\parskip}% 
  }% 
  \item[]}{\end{list}}



\begin{document}

\usebackgroundtemplate{\tikz\node[opacity=0.1]{\includegraphics[width=\paperwidth]{L05/960px-Delphi_Temple_of_Apollo.jpeg}};}
\begin{frame}
  \titlepage
  {\scriptsize
    Background: Temple of Apollo @ Delphi; by Skyring - Own work, CC BY-SA 4.0, \url{https://commons.wikimedia.org/w/index.php?curid=64170779}
    }
\end{frame}

\usebackgroundtemplate{}
\part{The Oracle Problem}
\begin{frame}
  \partpage
  
\end{frame}

\begin{frame}
  \frametitle{What's the right answer?}
  \Huge
  \begin{changemargin}{2em}
    cop-out: ``ask a human''
  \end{changemargin}
\end{frame}

\begin{frame}
  \frametitle{Begging the Question}
  \Large
  \begin{changemargin}{2em}
    Taking the answer the system computes \\ as the right answer.\\[1em]
    (is the basis for regression testing, though)
  \end{changemargin}
\end{frame}

\begin{frame}[fragile]
  \frametitle{Simple example: add}
  \Large
  \begin{changemargin}{2em}
    \begin{lstlisting}[language=Python]
      def add(x, y):
        return x + y
    \end{lstlisting}
    ~\\[1em]
    We all agree about the output of \texttt{add(1,1)}.\\[1em]

    Right?
  \end{changemargin}
\end{frame}

\begin{frame}[fragile]
  \frametitle{(Almost)}
  \Large
  \begin{changemargin}{2em}
    (What about \texttt{add("3", 5)} in JavaScript?)
  \end{changemargin}
\end{frame}

\begin{frame}[fragile]
  \frametitle{High school math}
  \Large
  \begin{changemargin}{2em}
    Consider function \texttt{solve\_quadratic()} for
    \[ x^2-2x-4 = 0. \]

    Need to read the function name and remember high school math.\\[2em]

    Also, edge cases: no solutions;\\ \qquad floating-point shenanigans.
  \end{changemargin}
\end{frame}

\begin{frame}
  \frametitle{Human Oracles: source of truth}
  \Large
  \begin{changemargin}{2em}
  Unit level: Mainly use the function name,\\
  plus any function documentation (if it exists).\\[1em]

  More generally: You use your human experience to say what the answer should be.
  \end{changemargin}
\end{frame}

\part{Helping Human Oracles}
\begin{frame}
  \partpage
\end{frame}

\begin{frame}
  \frametitle{Sometimes there is no alternative}
  \Huge
  \begin{changemargin}{2em}
    You may have to \\ ask a human.
  \end{changemargin}
\end{frame}

\begin{frame}
  \frametitle{Basis for judgment}
  \Large
  \begin{changemargin}{2em}
    Does the output meet the system requirements?\\[2em]

    (Eliciting requirements not in scope for this course.)
  \end{changemargin}
\end{frame}

\begin{frame}
  \frametitle{Easy versus hard inputs}
  \Large
  \begin{changemargin}{2em}
    Setting: function \texttt{calculate\_days\_between()}.\\[2em]
    What's the answer for:
    \begin{itemize}
    \item 12/24/2025 and 12/25/2025?
    \end{itemize}~ \\[0em]
    What about
    \begin{itemize}
    \item -5455/23195/-30879 and -5460/24100/-30800?
    \end{itemize}
    Even if we sanitize/declare negative numbers invalid, some
     inputs are still easier to check.
  \end{changemargin}
\end{frame}

\begin{frame}
  \frametitle{Input Profiles}
  \Large
  \begin{changemargin}{2em}
    Generate inputs that fit expected input profiles:\\[2em]
    
    Start with developers' sanity-check inputs \\
    \qquad (like 12/24/2025 and 12/25/2025 etc).\\[2em]

    Also, sanitizing checks inside the code are good places to start.\\[2em]

    Months: valid months, 0, -1, 13.\\[2em]
    
  \end{changemargin}

\end{frame}

\begin{frame}
  \frametitle{Other options}
  \Large
  \begin{changemargin}{2em}
    1. Start from normal inputs, \\
    \hspace*{2em} use genetic algorithms, \\
    \hspace*{2em} or generate from distributions.\\[1em]

    2. Reuse partial inputs, manually modified;\\
    \hspace*{2em} change one thing at a time,\\
    \hspace*{2em} thus, easier to reason about changes in the output.\\[1em]
    \hspace*{2em} e.g. go from 0/1/2010 to 1/1/2010, etc.
    
  \end{changemargin}

\end{frame}

\begin{frame}
  \frametitle{Integers vs strings}
  \large
  \begin{changemargin}{2em}
    Even though some integers aren't very good (e.g. 23195),\\
    it's still easier to create integers than strings,\\
    and easier to create strings than e.g. trees.\\[2em]

    Space of strings is bigger; \\
    space of sensible strings is proportionally smaller.\\[2em]

    Can use random strings as fuzzed inputs,\\
    but also want strings that pass sanity checks.\\[2em]

    Can mine the web for strings, or generate strings using heuristics
    (or LLMs).
        
  \end{changemargin}

\end{frame}

\begin{frame}
  \frametitle{Crowdsourcing inputs}
  \Large
  \begin{changemargin}{2em}
    People have tried to use Mechanical Turk.\\[2em]
    Apparently it's hard to get good results.
  \end{changemargin}
\end{frame}

\usebackgroundtemplate{\begin{tikzpicture}\node[opacity=0.3]{\hspace*{.2\paperwidth} \includegraphics[width=.6\paperwidth, trim=0 0 0 -20em]{L05/02936_mini_churchy_entrance_v1.jpg}};\end{tikzpicture}}
\begin{frame}
  \frametitle{Reducing volume of work for human testers}
  \Large
  \begin{changemargin}{2em}
    People always hope for test suite reduction.\\
    I'm not aware of good general-purpose solutions.\\[2em]
    Also: test case reduction; \\
    we'll talk about that in the fuzzing module.
  \end{changemargin}
\end{frame}

  
\end{document}
